\begin{mdframed}
    \textbf{La extensión máxima para esta sección es de 1 página.}
\end{mdframed}

En esta tarea, se implementaron y compararon dos algoritmos para calcular la distancia mínima de edición extendida entre dos cadenas: uno basado en fuerza bruta y otro en programación dinámica. Ambos algoritmos fueron diseñados para manejar costos variables en las operaciones de inserción, eliminación, sustitución y transposición, lo que añadió complejidad al problema y permitió evaluar su rendimiento en escenarios más realistas y aplicables en el mundo real.
El enfoque de fuerza bruta demostró ser adecuado para casos simples, permitiendo explorar todas las posibles secuencias de operaciones y garantizando la obtención de la solución óptima. Sin embargo, su complejidad exponencial lo hace inviable para cadenas de gran longitud, ya que el tiempo de ejecución aumenta drásticamente con el tamaño de las cadenas.
Por otro lado, el enfoque de programación dinámica resultó significativamente más eficiente en términos de tiempo de ejecución y uso de recursos, especialmente en casos de entrada de tamaño considerable. Este enfoque optimiza el cálculo almacenando soluciones de subproblemas intermedios, lo cual reduce los cálculos redundantes y permite una resolución más rápida y escalable del problema. La inclusión de transposiciones y costos variables se manejó de forma efectiva en la matriz de programación dinámica, lo cual permite que el algoritmo se adapte fácilmente a diferentes configuraciones de costos.
En resumen, los resultados muestran que la programación dinámica es el enfoque más adecuado para problemas de edición de cadenas con configuraciones complejas, debido a su balance entre precisión y eficiencia. Este trabajo destaca la importancia de seleccionar el paradigma de diseño de algoritmos adecuado según la naturaleza del problema, y aporta una base sólida para aplicaciones en las que es esencial minimizar los costos de edición en secuencias de datos.

