Primero, se cargan matrices y vectores de los costos, se abren los archivos con las palabras a trasnformar, y procesa cada par de palabras estas estando dividas por un delimitador ("|"). El cálculo del costo se realiza mediante un algoritmo que considera transposiciones, sustituciones, inserciones y eliminaciones, usando funciones auxiliares que manejan matrices y vectores. El diseño incluye manejo de errores, como fallas al abrir archivos o al procesar datos, y asegura que los resultados se guarden correctamente, proporcionando robustez y funcionalidad para su uso en análisis lingüísticos o similares.


La complejidad de este algoritmo es:
\begin{itemize}
\item Temporal: O(n*l), donde n es la cantidad de pares de palabras y l el largo promedio de ellas.
\item Espacial: O(L), donde L es el largo de la palabra más larga.
\end{itemize}
\begin{algorithm}[H]
    \SetKwProg{myproc}{Procedure}{}{}
    \SetKwFunction{Main}{main}
    \SetKwFunction{LoadMatrices}{cargarMatrices}
    \SetKwFunction{CalculateCost}{calcularCosto}

    \DontPrintSemicolon
    \footnotesize

    % Definición del procedimiento principal
    \myproc{\Main{}}{
        \LoadMatrices{}\;  % Cargar matrices iniciales necesarias para el cálculo
        archivo $\leftarrow$ abrir "palabras.txt" en modo lectura\;
        archivo\_salida $\leftarrow$ abrir "resultado.txt" en modo escritura\;

        \If{archivo no está abierto}{
            mostrar "No se pudo abrir el archivo."\;
            \Return 1\;  % Terminar el programa con código de error
        }
        \If{archivo\_salida no está abierto}{
            mostrar "No se pudo abrir el archivo de salida."\;
            \Return 1\;  % Terminar el programa con código de error
        }

        archivo $\rightarrow$ leer $N$\;
        archivo $\rightarrow$ ignorar hasta nueva línea\;

        \For{$i \leftarrow 0$ \KwTo $N-1$}{
            línea $\leftarrow$ leer línea desde archivo\;

            dividir línea en palabra1 y palabra2 usando '|' como delimitador\;
            
            costo1 $\leftarrow$ \CalculateCost{palabra1, palabra2}\;
            costo2 $\leftarrow$ \CalculateCost{palabra2, palabra1}\;
            costoMin $\leftarrow \min(costo1, costo2)$\;

            archivo\_salida $\rightarrow$ escribir $costo1$, $costo2$, $costoMin$\;
        }

        cerrar archivo\;
        cerrar archivo\_salida\;

        \Return 0\;  % Terminar el programa exitosamente
    }
    \caption{Estructura del algoritmo para procesar pares de palabras y calcular costos mínimos.}
    \label{alg:procesar_pares_palabras}
\end{algorithm}


\begin{algorithm}[H]
    \SetKwProg{myproc}{Procedure}{}{}
    \SetKwFunction{CalculateCost}{calcularCosto}
    \SetKwFunction{TransCost}{costo\_trans}
    \SetKwFunction{SubCost}{costo\_sub}
    \SetKwFunction{InsCost}{costo\_ins}
    \SetKwFunction{DelCost}{costo\_del}

    \DontPrintSemicolon
    \footnotesize

    % Definición de la función de cálculo de costo
    \myproc{\CalculateCost{palabra1, palabra2}}{
        costoTotal $\leftarrow 0$\;
        largoP1 $\leftarrow$ tamaño de palabra1\;
        largoP2 $\leftarrow$ tamaño de palabra2\;
        n $\leftarrow \min(largoP1, largoP2)$\;
        $i \leftarrow 0$\;

        \While{$i < n$}{
            \If{$i + 1 < n$ \textbf{y} palabra1[$i + 1$] = palabra2[$i$] \textbf{y} palabra1[$i$] = palabra2[$i + 1$]}{
                costoTotal $\leftarrow$ costoTotal $+$ \TransCost{palabra1[$i$], palabra1[$i + 1$]}\;
                $i \leftarrow i + 2$\;  % Avanzar el índice en dos posiciones
            }
            \Else{
                costoTotal $\leftarrow$ costoTotal $+$ \SubCost{palabra1[$i$], palabra2[$i$]}\;
                $i \leftarrow i + 1$\;  % Avanzar el índice en una posición
            }
        }

        % Costos de inserción o eliminación para caracteres restantes
        \For{$i \leftarrow i$ \KwTo largoP2 $- 1$}{
            costoTotal $\leftarrow$ costoTotal $+$ \InsCost{palabra2[$i$]}\;
        }
        \For{$i \leftarrow i$ \KwTo largoP1 $- 1$}{
            costoTotal $\leftarrow$ costoTotal $+$ \DelCost{palabra1[$i$]}\;
        }

        \Return costoTotal\;  % Retornar el costo total calculado
    }
    \caption{Algoritmo para calcular el costo entre dos palabras considerando sustituciones, transposiciones, inserciones y eliminaciones.}
    \label{alg:calcular_costo}
\end{algorithm}

\begin{algorithm}[H]
    \SetKwProg{myproc}{Procedure}{}{}
    \SetKwFunction{ReadMatrix}{leerMatriz}

    \DontPrintSemicolon
    \footnotesize

    % Definición de la función para leer la matriz
    \myproc{\ReadMatrix{nombreArchivo}}{
        archivo $\leftarrow$ abrir nombreArchivo en modo lectura\;
        matriz $\leftarrow$ matriz de $26 \times 26$ inicializada en ceros\;

        \If{archivo no está abierto}{
            mostrar "Error al abrir el archivo " + nombreArchivo\;
            \Return matriz\;  % Retornar matriz vacía en caso de error
        }

        \For{$i \leftarrow 0$ \KwTo $25$}{
            \For{$j \leftarrow 0$ \KwTo $25$}{
                archivo $\rightarrow$ leer matriz[$i$][$j$]\;
                \If{lectura falla}{
                    mostrar "Error al leer el archivo en la posición (" + $i$ + ", " + $j$ + ")"\;
                    cerrar archivo\;
                    \Return matriz\;  % Retornar matriz parcial en caso de error
                }
            }
        }

        cerrar archivo\;
        \Return matriz\;  % Retornar la matriz completa leída
    }
    \caption{Algoritmo para leer una matriz $26 \times 26$ desde un archivo.}
    \label{alg:leer_matriz}
\end{algorithm}


\begin{algorithm}[H]
    \SetKwProg{myproc}{Procedure}{}{}
    \SetKwFunction{ReadVector}{leerVector}

    \DontPrintSemicolon
    \footnotesize

    % Definición de la función para leer el vector
    \myproc{\ReadVector{nombreArchivo}}{
        archivo $\leftarrow$ abrir nombreArchivo en modo lectura\;
        
        \If{archivo no está abierto}{
            lanzar \textbf{error} "No se pudo abrir el archivo."\;
        }

        vector $\leftarrow$ vector de tamaño $26$\;

        \For{$i \leftarrow 0$ \KwTo $25$}{
            archivo $\rightarrow$ leer vector[$i$]\;
            \If{lectura falla}{
                lanzar \textbf{error} "Error al leer el archivo de vector."\;
            }
        }

        \If{tamaño de vector $\neq 26$}{
            lanzar \textbf{error} "El archivo no contiene exactamente 26 números."\;
        }

        cerrar archivo\;
        \Return vector\;  % Retornar el vector completo leido
    }
    \caption{Algoritmo para leer un vector de tamaño 26 desde un archivo, con manejo de errores.}
    \label{alg:leer_vector}
\end{algorithm}

Ejemplo paso a paso:
\begin{itemize}
\item1.	Se toman el par de palabras S1="ABAA" y S2="BAAA" y se empieza a leer letra a latra el S1.
\item2.	Se comprueba si S1[0]==S2[1] y si S1[1]==S2[0].
\item3. Si es así, se hace una trasposición y se avanza en 2 el lector de la palabra. En este caso si es igual por lo que hay transposición.
\item4. En caso contrario se hace una sustitución y se avanza en 1 el lector de la palabra.
\item5. Se repite hasta el fin de la palabra para posteriormente ajustar la palabra con ins o del segun corresponda. En este caso no es necesario.
\item6. Se repiten todos los pasos pero son S2 y se toma el costo mínimo. En este caso los costos quedan igual.
\end{itemize}